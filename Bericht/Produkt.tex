\section{Produkt}
%Produktbilder - Gesamtprodukt - Prototyping Moni
\subsection{zukünftiges Produktangebot}
Wir haben uns zwei Produktkategorien überlegt, mit denen wir Gewinn generieren können. Zum einen ist das ein monatliches Bezahlkonzept. Hierbei haben wir uns an dem Konzept einiger Mobilfunkanbieter orientiert. Die Überlegung ist, dass wir zwei hinzu buchbare Optionen anbieten und man nur das bezahlt, was wirklich gebraucht wird. Eine Optionen ist die Google Standort API für eine genauere und zuverlässigere Standorterkennung. Die andere Option ist das Bestellen einer Sim-Karte mit Mobilfunkvertrag. Die Idee ist hier, dass so gegen einen kleinen Aufpreis der Ease-of-Use unseres Produkts weiter gesteigert wird.

Für das Endgerät haben wir uns neben der aktuellen Basis Variante für eine Base+ und eine Pro Variante entschieden. Bei der Base+ Variante ergänzen wir die Basis Variante lediglich um eine Sirene und etwas mehr Akkuleistung. Für die Pro Variante erweitern wir das Basismodell um einen WLAN Empfänger. Des Weiteren überlegen wir unseren modularen Aufbau für eine weitere Umsatzsteigerung zu nutzen und Akkumodule zu verkaufen, mit denen die Akkulaufzeit unserer Produkte vom Nutzer beliebig verlängert werden kann. Ein weiterer Vorteil ist, dass so bei nachlassender Akkuleistung unser Produkt nicht entsorgt werden muss, sondern der Nutzer einfach mit offiziellen Produkten den Akku wechseln kann. Dies soll verhindern, dass der Kunde auf Drittanbieter zurückgreift.
Das Ziel ist, dass wir unsere Apps und den Webserver unabhängig von der Produktversion verwenden können.

\subsection{Vergleich} Moni