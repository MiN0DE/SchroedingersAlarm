\section{Ideensammlung}
\subsection{Starkstrom Taser Auto} Moni
\subsection{Autoleere Stadt}
Eine unserer Ideen war es, anstatt eine direkte Lösung für das Problem zu finden, das Problem selbst zu beseitigen. Hierfür haben wir uns vorgestellt, dass wir einfach Autos generell abschaffen. Die Idee ist, dass wenn es keine Autos mehr gibt, können sie auch nicht mehr geklaut werden. Diese Idee ist jedoch an der Umsetzbarkeit gescheitert. Es gibt zwar bereits Innenstädte, wo Autos verboten sind, jedoch ist dies nicht gleichzusetzen mit dem kommpletten Abschaffen von Autos, da diese zumeist nur außerhalb der Stadt geparkt sind. Die Idee wurde letztendlich verworfen, da die Umsetzung dieses Projekt unseren Projektrahmen gesprengt hätte. 
\subsection{Gesichtserkennung und Startunterbrechung}
Bei dieser Lösung hatten wir uns überlegt, mittels Gesichtserkennung zu überprüfen, ob die sich auf dem Fahrersitz befindliche Person der Besitzer oder ein berechtigter Fahrer des Fahrzeugs ist. Die Zündung des Autos soll hier nur dann möglich sein, wenn der Fahrer erkannt wird. Das Problem an dieser Modifikation und der Grund, warum wir uns gegen diese Idee entschieden haben, ist die Zulassungspflicht durch den TÜV. Um den Motor am Starten zu hindern, müssten wir uns entweder auf das Steuergerät des Autos mit aufschalten oder direkt eine Fernsteuerung für die Motorklemme mit einbauen. Die erste Möglichkeit scheitert daran, dass sich der Zugang zu den Steuersystemen des Autos schwer gestaltet, da je nach Modell die Steuersoftware unterschiedlich ist, womit allgemeine Ansätze ausgeschlossen sind. Des Weiteren erlauben Hersteller entweder generell nicht, dass Drittanbieter ihre Software nutzen oder stimmen nur gegen hohe Lizenzgebühren einer Nutzung zu. Ein weiteres Problem ist, dass beide Lösungen eine professionelle Installation bedürfen, was unserem Konzept widerspricht. Abgesehen davon ist es wie, bereits erwähnt, notwendig, für jedes Fahrzeugmodell die entsprechende Straßenzulassung für die Modifikation zu erhalten. Entsprechend ist der Arbeitsaufwand für unseren Projektzeitraum zu groß. 